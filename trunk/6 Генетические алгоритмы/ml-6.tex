\documentclass[14pt, fleqn, xcolor={dvipsnames, table}]{beamer}
\usepackage[T2A]{fontenc}
\usepackage[utf8]{inputenc}
\usepackage[english,russian]{babel}
\usepackage{amssymb,amsfonts,amsmath,mathtext}
\usepackage{cite,enumerate,float,indentfirst}
\usepackage{cancel}

\usepackage{tikz}                   
\usetikzlibrary{shadows}

% \usepackage{enumitem}
% \setitemize{label=\usebeamerfont*{itemize item}%
%   \usebeamercolor[fg]{itemize item}
%   \usebeamertemplate{itemize item}}

\graphicspath{{images/}}

\usetheme{Madrid}
\usecolortheme{seahorse}
\renewcommand{\CancelColor}{\color{red}}

\setbeamercolor{footline}{fg=Blue!50}
\setbeamertemplate{footline}{
  \leavevmode%
  \hbox{%
  \begin{beamercolorbox}[wd=.333333\paperwidth,ht=2.25ex,dp=1ex,center]{}%
    И. Кураленок, Н. Поваров, Яндекс
  \end{beamercolorbox}%
  \begin{beamercolorbox}[wd=.333333\paperwidth,ht=2.25ex,dp=1ex,center]{}%
    Санкт-Петербург, 2013
  \end{beamercolorbox}%
  \begin{beamercolorbox}[wd=.333333\paperwidth,ht=2.25ex,dp=1ex,right]{}%
  Стр. \insertframenumber{} из \inserttotalframenumber \hspace*{2ex}
  \end{beamercolorbox}}%
  \vskip0pt%
}
\newcommand\indentdisplays[1]{%
     \everydisplay{\addtolength\displayindent{#1}%
     \addtolength\displaywidth{-#1}}}
\newcommand{\itemi}{\item[\checkmark]}

\title{Генетические алгоритмы\\\small{}}
\author[]{\small{%
И.~Куралёнок,
Н.~Поваров}}
\date{}

\begin{document}

\begin{frame}
\maketitle
\small
\begin{center}
\vspace{-60pt}
\normalsize {\color{red}Я}ндекс \\
\vspace{80pt}
\footnotesize СПб, 2013
\end{center}
\end{frame}

\section{Содержание}
\begin{frame}{Содержание}
\begin{enumerate}
  \item Генетические алгоритмы
  \item Свойства генетических алгоритмов
  \item Differential evolution
  \item Теорема об отсутствии халявы
\end{enumerate}
\end{frame}

\begin{frame}{Идея}
Кажется природа решает ту же задачу. \\
$\Rightarrow$ Можно подсмотреть механизмы \\
$\Rightarrow$ Популяция позволяет говорить о "производных" \\
$\Rightarrow$ Внутри популяции можно исследовать зависимости и использовать их \\
$\Rightarrow$ Просто объяснять, но нужен специальный "масонский" язык \\
Отсебятена: генетические алгоритмы скорее язык описания перебора.
\end{frame}

\begin{frame}{Этапы генетического алгоритма}
\begin{enumerate}
  \item Выбрать начальную популяцию
  \item Измерить "приспособленность" особей
  \item Повторить до сходимости
  \begin{enumerate}
    \item Выбрать представителей популяции для размножения
    \item Породить новых особей с помощью скрещивания и мутаций
    \item Измерить приспособленность "детей"
    \item Заменить наименее приспособленных "детьми"
  \end{enumerate}
\end{enumerate}
\end{frame}

\begin{frame}{Выбор начальной позиции}
Как любые переборные методы генетика существенно зависима от начальных позиций. \\
Можно так:
\begin{itemize}
  \item Сводят пространство решений в $[0,1]^n$ и берут равномерно рандомные точки
  \item Сводят к ${0,1}^k$ с помощью бинаризации и берут орты
  \item Делают предварительное сэмплирование
\end{itemize}
\end{frame}

\begin{frame}{Выбор особей для размножения}
Выбор особей - ключевой момент! \\
Можно так:
\begin{itemize}
  \item Выбирать с вероятностью, пропорциональной "приспособленности"
  \item Предварительно делать shuffle "приспособленности"
  \item От "приспособленности" зависит очередность
  \item Можно выбирать равномерно из популяции (по классике нельзя)
  \item Любые комбинации :)
\end{itemize}
\end{frame}

\section{No free lunch theorem}
\begin{frame}{Постановка вопроса}
\textit{Целых 2 занятия мы обходим пространство решений. Может быть есть лучший из всех методов обхода?}
В качестве аналогии:
\begin{description}
  \item[Ресторан:] способ обхода пространства решений
  \item[Блюдо:] конкретная задача, требующая оптимизации
  \item[Цена:] значение целевой функции на обойденной части пространства
\end{description}
\begin{theorem}[No free lunch theorem]
Все рестораны в среднем одинаково дороги
\end{theorem}
\end{frame}

\begin{frame}{NFL: формальная формулировка}
$$\begin{array}{rl}
d_m =& \{(d^x_m(1), d^y_m(1)), \ldots, (d^x_m(m), d^y_m(m))\} \\
f: & \mathcal{X} \to \mathcal{Y} \\
\mathcal{F}= & \mathcal{Y}^\mathcal{X} \\
p(f) = & \frac{1}{\mathcal{F}} ёё
\end{array}$$

\begin{theorem}[David Wolpert and William G. Macready (1997)]
Для любых двух способов обхода $a_1$ и $a_2$:
$$\begin{array}{rl}
\sum_f p(d^y_m|f,m,a_1) = \sum_f p(d^y_m|f,m,a_2) \\
\sum_f p(d^y_m|f_0,M,m,a_1) = \sum_f p(d^y_m|f_0,M,m,a_2)
\end{array}$$
\end{theorem}
\end{frame}

\begin{frame}{NFL: следствия}
\begin{itemize}
  \item Нам всем хватит работы :)
  \item Полоса белая, полоса черная
  \item Нужно искать близкие задачи
\end{itemize}
\end{frame}

\end{document}
