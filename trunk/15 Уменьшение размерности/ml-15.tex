\documentclass[14pt, fleqn, xcolor={dvipsnames, table}]{beamer}
\usepackage[T2A]{fontenc}
\usepackage[utf8]{inputenc}
\usepackage[english,russian]{babel}
\usepackage{amssymb,amsfonts,amsmath,mathtext}
\usepackage{cite,enumerate,float,indentfirst}
\usepackage{cancel}
\usepackage{graphicx}
\usepackage{animate}

\usepackage{tikz}
% \usepackage{enumitem}
\usetikzlibrary{shadows}

% \usepackage{enumitem}
% \setitemize{label=\usebeamerfont*{itemize item}%
%   \usebeamercolor[fg]{itemize item}
%   \usebeamertemplate{itemize item}}

\graphicspath{{images/}}

\usetheme{Madrid}
\usecolortheme{seahorse}
\renewcommand{\CancelColor}{\color{red}}

\setbeamercolor{footline}{fg=Blue!50}
\setbeamertemplate{footline}{
  \leavevmode%
  \hbox{%
  \begin{beamercolorbox}[wd=.333333\paperwidth,ht=2.25ex,dp=1ex,center]{}%
    И. Кураленок, Н. Поваров, Яндекс
  \end{beamercolorbox}%
  \begin{beamercolorbox}[wd=.333333\paperwidth,ht=2.25ex,dp=1ex,center]{}%
    Санкт-Петербург, 2014
  \end{beamercolorbox}%
  \begin{beamercolorbox}[wd=.333333\paperwidth,ht=2.25ex,dp=1ex,right]{}%
  Стр. \insertframenumber{} из \inserttotalframenumber \hspace*{2ex}
  \end{beamercolorbox}}%
  \vskip0pt%
}
\newcommand\indentdisplays[1]{%
     \everydisplay{\addtolength\displayindent{#1}%
     \addtolength\displaywidth{-#1}}}
\newcommand{\itemi}{\item[\checkmark]}

\newenvironment{mydescription}[1]
  {\begin{list}{}%
   {\renewcommand\makelabel[1]{\color{blue}##1:\hfill}%
   \settowidth\labelwidth{\makelabel{#1}}%
   \setlength\leftmargin{\labelwidth}
   \addtolength\leftmargin{\labelsep}}}
  {\end{list}}

\title{Уменшение размерности: обзор\\\small{}}
\author[]{\small{%
И.~Куралёнок,
Н.~Поваров}}
\date{}
\begin{document}

\begin{frame}
\maketitle
\small
\begin{center}
\vspace{-60pt}
\normalsize {\color{red}Я}ндекс \\
\vspace{80pt}
\footnotesize СПб, 2014
\end{center}
\end{frame}

\section{Содержание}
\section{Постановка задачи уменьшения размерности}
\subsection{Зачем нужно}
\begin{frame}{Зачем бороться с размерностью?}
\begin{enumerate}
  \item Наглядность: сложно смотреть на 100-мерное пространство
  \item Количество данных и надежно подбираемых параметров зависимы
  \item Экономим стоимость использования:
  \begin{itemize}
    \item вычислительные ресурсы на этапе обучения и/или использования;
    \item стоимость отдельного измерения.
  \end{itemize}
  \item Убрать шум
  \item Гибкость в построении данных
\end{enumerate}
\end{frame}

\subsection{Какие фичи бывают}
\begin{frame}{Постановка задачи обучения}
\begin{itemize}
  \item С построением фичей (повесим на клиента датчик) $\Rightarrow$ хотим обвесить датчиками по самые не балуй
  \item Без построения фичей (льется поток неведомых данных) $\Rightarrow$ боимся потерять ценный сигнал, так как непонятно как его отделить от неценного
\end{itemize}
\end{frame}

\subsection{Почему можно} % тут можно привести аналогию с каналом
\begin{frame}{Почему можно в конструктиве}
\end{frame}

\begin{frame}{Почему можно в raw}
\end{frame}

\subsection{Направления}
\begin{frame}{Как можно это делать}
\begin{itemize}
  \item Feature selection + причины (whitening)
  \item Feature extraction (construction) + причины (kostyl production)
  \item Embedded models (velosiped prouction)
\end{itemize}
\end{frame}

\begin{frame}{Область применения}
\begin{itemize}
  \item Feature selection: 
  \begin{itemize}
    \item шум/сигнал (dB) у некоторых совсем поганый
    \item оценка источника сигнала
  \end{itemize}
  \item Feature extraction (construction)
  \begin{itemize}
    \item Уменьшить шум за счет ``похожести'' измерений про одно и то же
    \item Уменьшить стоимость на этапе обучения и/или использования (предобработка)
    \item Посмотреть на данные
    \item Структурирование обучения
  \end{itemize}
  \item Embedded models
  \begin{itemize}
    \item Когда не хочется думать

    \item Когда хочется рассуждать в терминах получаемого решения, а не в терминах данных (моделька красивая -- главная цель)
  \end{itemize}
\end{itemize}
\end{frame}

\subsection{Какие есть теоретические ограничения}
\begin{frame}{Jonson-Lindestrauss lemma}
\end{frame}

\section{Feature selection}
\subsection{Виды FS}
\begin{frame}{Как можно делать feature selection}
\begin{itemize}
  \item Wrapper
  \item Filter
  \item Embedded
\end{itemize}
\end{frame}

\subsection{Одиночный FS}

\subsection{Optimal subset selection}

\end{document}
