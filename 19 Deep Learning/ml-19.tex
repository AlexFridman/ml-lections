\documentclass[14pt, fleqn, xcolor={dvipsnames, table}]{beamer}
\usepackage[T2A]{fontenc}
\usepackage[utf8]{inputenc}
\usepackage[english,russian]{babel}
\usepackage{amssymb,amsfonts,amsmath,mathtext}
\usepackage{cite,enumerate,float,indentfirst}
\usepackage{cancel}
\usepackage{graphicx}
\usepackage{animate}

\usepackage{tikz}
% \usepackage{enumitem}
\usetikzlibrary{shadows}

% \usepackage{enumitem}
% \setitemize{label=\usebeamerfont*{itemize item}%
%   \usebeamercolor[fg]{itemize item}
%   \usebeamertemplate{itemize item}}

\graphicspath{{images/}}

\usetheme{Madrid}
\usecolortheme{seahorse}
\renewcommand{\CancelColor}{\color{red}}

\setbeamercolor{footline}{fg=Blue!50}
\setbeamertemplate{footline}{
  \leavevmode%
  \hbox{%
  \begin{beamercolorbox}[wd=.333333\paperwidth,ht=2.25ex,dp=1ex,center]{}%
    И. Кураленок, Н. Поваров, Яндекс
  \end{beamercolorbox}%
  \begin{beamercolorbox}[wd=.333333\paperwidth,ht=2.25ex,dp=1ex,center]{}%
    Санкт-Петербург, 2014
  \end{beamercolorbox}%
  \begin{beamercolorbox}[wd=.333333\paperwidth,ht=2.25ex,dp=1ex,right]{}%
  Стр. \insertframenumber{} из \inserttotalframenumber \hspace*{2ex}
  \end{beamercolorbox}}%
  \vskip0pt%
}
\newcommand\indentdisplays[1]{%
     \everydisplay{\addtolength\displayindent{#1}%
     \addtolength\displaywidth{-#1}}}
\newcommand{\itemi}{\item[\checkmark]}

\newenvironment{mydescription}[1]
  {\begin{list}{}%  
   {\renewcommand\makelabel[1]{\color{blue}##1:\hfill}%
   \settowidth\labelwidth{\makelabel{#1}}%
   \setlength\leftmargin{\labelwidth}
   \addtolength\leftmargin{\labelsep}}}
  {\end{list}}

\title{Деревья решений\\\small{}}
\author[]{\small{%
И.~Куралёнок,
Н.~Поваров}}
\date{}
\begin{document}

\begin{frame}
\maketitle
\small
\begin{center}
\vspace{-60pt}
\normalsize {\color{red}Я}ндекс \\
\vspace{80pt}
\footnotesize СПб, 2014
\end{center}
\end{frame}

\section{Определение и пример}
\begin{frame}{Пример}
Дмитрий и Владимир выбирают страну, куда поехать на отдых. Надо им помочь.
\end{frame}

\begin{frame}{Свойства}
\begin{itemize}
  \item Кусочно-постоянная природа
  \item Большая изменчивость при изменении learn
\end{itemize}
\end{frame}

\section{Алгоритм CART}

\section{Почему это может работать}
\subsection{Жадность}
\subsection{Количество информации в принятии решения}
\section{Варианты чувств прекрасного}
\begin{frame}{Варианты чувства прекрасного}
\begin{itemize}
  \item Энтропия
  \item Gini index
  \item AIC/BIC/etc. 
\end{itemize}
\end{frame}

\begin{frame}{Диалоги о дисперсии}
В случае MSE минимизируем суммарную дисперсию на каждом сплите.

Дисперсия растет с числом точек, нам интересно, что будет на бесконечности. Можно использовать исправленную дисперсию, но помогает не всегда и есть варианты.

\end{frame}
\begin{frame}{Диалоги о ожидании} % Stein paradox
Только что мы поняли, что среднее отклонение по выборке работает плохо. А как же быть со средним выборочным в качестве мат ожидания. \\
Проведем такой эксперимент: возьмем случайный нормальный вектор с единичной матрицей ковариаций и ненулевым, сравнимым с 1 ожиданием $\mu$. Хотим найти $\hat{\theta}: \min \|\theta - \mu\|$. В этом случае очевидное решение $\hat{\theta} = \frac{1}{n}\sum x_i$, окажется неоптимальным. % см код DispersionTest
\end{frame}

\begin{frame}{Диалоги о ожидании} % Stein paradox
Нам парадокс Штайна интересен в двух аспектах:
\begin{enumerate}
  \item Вычисляя среднее в листе, мы можем считать, что у нас есть 1 реализация вектора размерности $n$ и применить магию js-поправки.
  \item Принадлежность точки к ноду можно считать подругому: $\sum_i I\{x_{f(i)} > c_i\} > l$, где $f(i)$ --- номер фактора, использованного при делении на $i$-м уровне, $c_i$ --- условие деления. Сама сумма может быть рассмотрена как сумма бернуллевских случайных величин в условиях данной точки и к ней можно попробовать применить поправку Штайна.
\end{enumerate}
\end{frame}

\begin{frame}{Диалоги о ожидании} % Stein paradox
Все было бы хорошо, если бы:
\begin{itemize}
  \item Поправка была бы устойчива к ошибкам; % см test1FoldErr10 где js сливает наиву в одну калитку
  \item Мы хорошо бы знали дисперсию и форму распределения.
\end{itemize}
\end{frame}


\section{Пара трюков}
\subsection{Гистограммы}
\subsection{Разложение целевой функции в ряд}

\begin{frame}{Что мы сегодня узнали}
\begin{itemize}
  \item Деревья решений = разделяй и влавствуй
  \item Жадность + ограничение информации в делении = Profit
  \item Есть много чувств прекрасного, и хорошее еще не изобретено
  \item Пара трюков для ускорения процесса вычисления
\end{itemize}
\end{frame}
\end{document}
