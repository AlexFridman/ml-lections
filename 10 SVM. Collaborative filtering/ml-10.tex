\documentclass[14pt, fleqn, xcolor={dvipsnames, table}]{beamer}
\usepackage[T2A]{fontenc}
\usepackage[utf8]{inputenc}
\usepackage[english,russian]{babel}
\usepackage{amssymb,amsfonts,amsmath,mathtext}
\usepackage{cite,enumerate,float,indentfirst}
\usepackage{cancel}

\usepackage{tikz}                   
\usetikzlibrary{shadows}

% \usepackage{enumitem}
% \setitemize{label=\usebeamerfont*{itemize item}%
%   \usebeamercolor[fg]{itemize item}
%   \usebeamertemplate{itemize item}}

\graphicspath{{images/}}

\usetheme{Madrid}
\usecolortheme{seahorse}
\renewcommand{\CancelColor}{\color{red}}

\setbeamercolor{footline}{fg=Blue!50}
\setbeamertemplate{footline}{
  \leavevmode%
  \hbox{%
  \begin{beamercolorbox}[wd=.333333\paperwidth,ht=2.25ex,dp=1ex,center]{}%
    И. Кураленок, Н. Поваров, Яндекс
  \end{beamercolorbox}%
  \begin{beamercolorbox}[wd=.333333\paperwidth,ht=2.25ex,dp=1ex,center]{}%
    Санкт-Петербург, 2013
  \end{beamercolorbox}%
  \begin{beamercolorbox}[wd=.333333\paperwidth,ht=2.25ex,dp=1ex,right]{}%
  Стр. \insertframenumber{} из \inserttotalframenumber \hspace*{2ex}
  \end{beamercolorbox}}%
  \vskip0pt%
}
\newcommand\indentdisplays[1]{%
     \everydisplay{\addtolength\displayindent{#1}%
     \addtolength\displaywidth{-#1}}}
\newcommand{\itemi}{\item[\checkmark]}

\title{Линейные модели: SVM (продолжение). Collaborative filtering.\\\small{}}
\author[]{\small{%
И.~Куралёнок,
Н.~Поваров}}
\date{}

\begin{document}

\begin{frame}

\maketitle
\small
\begin{center}
\vspace{-60pt}
\normalsize {\color{red}Я}ндекс \\
\vspace{80pt}
\footnotesize СПб, 2013
\end{center}
\end{frame}

\section{SVM (продолжение)}
\subsection{SVM в случае неразделимых множеств} % из вики
\subsection{Сведение SVM к линейной системе с регуляризацией} % из книжки
\subsection{Простая оптимизация SVM} % ? из головы /machinelearning.ru
\section{Построение мультиклассификатора} % best 10 years paper award ICML2010
\section{Collaborative filtering}
\subsection{Пример} % переработать http://habrahabr.ru/company/surfingbird/blog/139022/ на Сифона/Бороду
\subsection{Постановка задачи} % wiki
\subsection{Случай, когда матрица полная} % Воронцов, лекция в ВШЭ http://www.machinelearning.ru/wiki/images/e/e5/Voron-2008-11-10-cf.pdf
\subsection{Разреженные матрицы: решение факторизацией} %  из головы
\subsection{Простое SVD разложение}  % Гантмахер
\subsection{Известные регуляризации} % Мучать Гулина
\subsection{Factorization machines} % Статья http://www.ismll.uni-hildesheim.de/pub/pdfs/Rendle2010FM.pdf
\end{document}
