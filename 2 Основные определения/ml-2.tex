\documentclass[14pt, fleqn, xcolor={dvipsnames, table}, hyperref={unicode}, babel={english,russian}, inputenc=utf8x]{beamer}

% \usepackage{mathtext}
% \usepackage[unicode]{hyperref}
% \usepackage[utf8x]{inputenc}
\usepackage[T2A]{fontenc}
\usepackage[utf8x]{inputenc}
\usepackage[russian,english]{babel}
\usepackage{amsmath,amsthm,amssymb}
% \selectlanguage{russian}


\usepackage{cite,enumerate,float,indentfirst}
\usepackage{csvsimple}
\usepackage{datatool}

\usepackage{tikz}                   
\usetikzlibrary{shadows}

% \usepackage{enumitem}
% \setitemize{label=\usebeamerfont*{itemize item}%
%   \usebeamercolor[fg]{itemize item}
%   \usebeamertemplate{itemize item}}

\graphicspath{{images/}}

\usetheme{Madrid}
\usecolortheme{seahorse}

\setbeamercolor{footline}{fg=Blue!50}
\setbeamertemplate{footline}{
  \leavevmode%
  \hbox{%
  \begin{beamercolorbox}[wd=.333333\paperwidth,ht=2.25ex,dp=1ex,center]{}%
    И. Кураленок, Н. Поваров, Яндекс
  \end{beamercolorbox}%
  \begin{beamercolorbox}[wd=.333333\paperwidth,ht=2.25ex,dp=1ex,center]{}%
    Санкт-Петербург, 2013
  \end{beamercolorbox}%
  \begin{beamercolorbox}[wd=.333333\paperwidth,ht=2.25ex,dp=1ex,right]{}%
  Стр. \insertframenumber{} из \inserttotalframenumber \hspace*{2ex}
  \end{beamercolorbox}}%
  \vskip0pt%
}
\newcommand\indentdisplays[1]{%
     \everydisplay{\addtolength\displayindent{#1}%
     \addtolength\displaywidth{-#1}}}
\newcommand{\itemi}{\item[\checkmark]}

\title{Машинное обучение: начало\\\small{}}
\author[]{\small{%
И.~Куралёнок,
Н.~Поваров}}
\date{}

\begin{document}

\begin{frame}
\maketitle
\small
\begin{center}
\vspace{-60pt}
\normalsize {\color{red}Я}ндекс \\
\vspace{80pt}
\footnotesize СПб, 2013
\end{center}
\end{frame}

\section{Постановка задачи}
\begin{frame}
\frametitle{Задача на сегодня}
\begin{description}
\leftmargin=-5em
\itemindent=0em
\labelwidth=0em
\leftskip=-5em
\item[Задача:] отделить ``хороших'' студентов от ``плохих''
\item
\item[Формально:] предсказать средний балл следующей сессии
\end{description}
\end{frame}

\begin{frame}
\frametitle{План работы}
\begin{enumerate}
\item Определиться с тем кто такой студент
\item Как из этого определения можно понять хороший он или нет
\item Выработать чувство прекрасного
\item Решить полученную задачу оптимизации
\item Оценить качество полученного решения
\item Проанализировать результаты
\end{enumerate}
\end{frame}

\section{Векторизация и факторы}
\begin{frame}
\frametitle{Векторизация\footnote{В данном случае не про графику.}}
\begin{description}
\leftmargin=-5em
\itemindent=0em
\labelwidth=0em
\leftskip=-5em
\item[Векторизация:] перевод представления о предмете в векторное выражение.
\item[]
\item[] Компоненты полученного в результате векторизации вектора будем называть {\bfфакторами}
\end{description}
\end{frame}

\section{Векторизация и факторы}
\begin{frame}
\frametitle{Факторы про студента}
\small
\begin{center}
\begin{tabular}{p{0.45\textwidth}p{0.5\textwidth}}
\begin{itemize}
  \item Пол
  \item Средний школьный балл
  \item Школа номер
  \item Школа город
  \item Доля пропущенных лекций
  \item Оценка по мнению родителей
\end{itemize} &
\begin{itemize}
  \item Пиво/неделя
  \item Друзей в ОК/FB/ВК
  \item Расстояние от дома до универа
  \item Ряд в аудитории
  \item Наличие планшета
  \item Периметр головы
\end{itemize}
\end{tabular}
\end{center}
\end{frame}
\begin{frame}[t]\frametitle{Множество на котором обучаемся $L$}
\tiny
    \DTLsetseparator{,}
    \renewcommand\dtlstringalign{p{4.5em}}
    \renewcommand\dtlrealalign{p{1.5em}}
    \renewcommand\dtlintalign{p{1.5em}}
    \renewcommand\dtldisplayvalign{c}
    \renewcommand\dtldisplayafterhead{\hline\\}
    \renewcommand\dtlheaderformat[1]{\textbf{#1}}
    \DTLloaddb[noheader,%
    headers={Пол,СШБ,Город,Тип Шк,Проп.,Род.,Пиво,Друзей,D,Ряд,iPad,Голова,Баллы}%
    ]{data}{sample_set_filtered.csv}
    
    \DTLdisplaydb{data}
\end{frame}

\section{Решающая и целевая функции}
\begin{frame}[t]\frametitle{Решающая функция}
    $$\begin{array}{c}
      h_{w_0}(x) = w_0^{T}x \\
      \\
      w_0, x \in \mathbb{R}^n \\
      \end{array}
    $$
\begin{itemize}
  \item Простая
  \item Универсальная
  \item Легко интерпретируемая
\end{itemize}
\end{frame}

\section{Нормализация факторов}
\begin{frame}{Адаптация факторов}
Не все факторы подходят для использования в выбранной модели:
\begin{itemize}
  \item Город -- плохой фактор, не из $\mathbb{R}$, $\Rightarrow$ местные/неместные $\in \{0,1\}$
  \item Тип школы $\Rightarrow$
  \begin{enumerate}
    \item Школа = Лицей/Гимназия $\in \{0,1\}$
    \item Школа = 239 $\in \{0,1\}$
  \end{enumerate}
  \item Оценка по мнению родителей -- есть пропущенные значения, $\Rightarrow$ хуже/неизвестно/лучше чем было в школе $\in \{-1,0,1\}$
  \item Планшеты есть у всех $\Rightarrow$ выкинули
\end{itemize}
\end{frame}

\begin{frame}{Итоговый вектор факторов}
\begin{center}
\footnotesize
\begin{tabular}{rlrl}
1 & Пол & 2 & Средний школьный балл \\
3 & школа = 239 & 4 & Понаехали \\
5 & Гимназия/Лицей & 6 & Доля пропущенных лекций \\
7 & Оценка по мнению родителей & 8 & Пиво/неделя \\
9 & Друзей в ОК/FB/ВК & 10 & Расстояние от дома до универа \\
11 & Ряд в аудитории & 12 & Периметр головы
\end{tabular}
\end{center}
\end{frame}

\begin{frame}[t]\frametitle{Решающая функция}
    $$\begin{array}{c}
      h_{w_0,b_0}(x) = w_0^{T}x + b_0 \\
      \\
      w_0, x \in \mathbb{R}^n, b_0 \in \mathbb{R} \\
      \end{array}
    $$
\begin{itemize}
  \item Простая
  \item Универсальная
  \item Легко интерпретируемая
\end{itemize}
\end{frame}


\begin{frame}[t]\frametitle{Целевая функция}
    $$
      (w_0,b_0) = \arg\min_{(w,b)} \sum_{(x_i,y_i) \in L} \| h_{w,b}(x_i) - y_i \|
    $$
\begin{itemize}
  \item MSE --- годное первое приближение в $\mathbb{R}$
  \item Интерпретируемое значение $\min$
\end{itemize}
\end{frame}

\begin{frame}[t]\frametitle{Решение и его интерпретация}
    $$
      \begin{array}{ll}
      (w_0, b_0) &= \arg\min_w \sum_{(x_i,y_i) \in L} \left\| x_i^Tw + b - y_i \right\| \\
       &= \arg\min_w \left\|X^Tw + b - y\right\|^2 \\
       &\left\{\begin{array}{l}
        w_0 = \left(XX^T\right)^{-1}X(y - 1b) \\
        b_0 = \frac{1}{n} \sum_{i=1}^n y_i
       \end{array}\right.
      \end{array}
    $$
    Вот что получилось:
    {\scriptsize
    $$\begin{array}{ll}
      w_0 &= -0.24, 1.17, -5.82, 0.32, 0.30, -2.33, 1.77, -0.01, -0.00, 0.13, 0.03, -0.08\\
      b_0 &= 17.04 \\
      T&= 83.06 \\
      \sqrt{\frac{T(w_0, b_0)}{n}} &= 1.72\\
    \end{array}
    $$}
\end{frame}

\begin{frame}{Интерпретация результата}
Обещали, что результат хорошо интерпретируется. Влияние факторов:
\footnotesize
\begin{description}
\item[Школа = 239 (3), вес $=-5.82$] Вот такая вот выборка :)
\item[Доля пропущенных лекций (6), вес $=-2.33$] на лекции надо ходить
\item[Оценка по мнению родителей (7), вес $=1.77$] честность -- это хорошо :)
\end{description}
\end{frame}

\begin{frame}{Интерпретация результата}
Обещали, что результат хорошо интерпретируется. Влияние факторов:
\footnotesize
\begin{description}
\item[Школа = 239 (3), вес $=-5.82$] вот такая вот выборка :)
\item[Доля пропущенных лекций (6), вес $=-2.33$] на лекции надо ходить
\item[Оценка по мнению родителей (7), вес $=1.77$] честность -- это хорошо :)
\end{description}
\center{\textbf{\Large Все это фигня!}}
\end{frame}

\begin{frame}{Нормализация факторов}
Хотим простого: $0$ мат. ожидание и $1$ дисперсию.
$$
\begin{array}{ll}
y_i=\frac{x_i-m_i}{\sigma_i}
\end{array}
$$

{\em Это стандартная техника, но немного (2 слайда) подумайте насколько она нам подходит.}

{\scriptsize $$\begin{array}{ll}
  w_0 &= -0.16, 0.74, -1.47, 0.19, 0.16, -0.67, 1.19, -0.05, -0.02, 0.27, 0.11, -0.08\\
      b_0 &= 17.04 \\
      T(w_0, b_0) &= 82.72 \\
      \sqrt{\frac{T(w_0, b_0)}{n}} &= 1.72 \\
  \end{array}
$$}
\end{frame}

\begin{frame}{Интерпретация результата II}
Обещали, что результат хорошо интерпретируется. Влияние факторов:
\footnotesize
\begin{description}
\itemindent=0em
\labelwidth=0em
\leftskip=-5em  
\item[Школа = 239 (3), вес $=-1.47$] таки да :)
\item[Оценка по мнению родителей (7), вес $=1.19$] честность еще важнее!
\item[Средний школьный балл (2), вес $=0.74$] багаж знаний -- определяет (догадайтесь какой курс?)
\item[Доля пропущенных лекций (6), вес $=-0.67$] опустилась вниз
\end{description}
\end{frame}

\begin{frame}{Интерпретация результата II}
Обещали, что результат хорошо интерпретируется. Влияние факторов:
\footnotesize
\begin{description}
\itemindent=0em
\labelwidth=0em
\leftskip=-5em  
\item[Школа $=239$ (3), вес $=-1.47$] Таки да :)
\item[Оценка по мнению родителей (7), вес $=1.19$] честность еще важнее!
\item[Средний школьный балл (2), вес $=0.74$] багаж знаний -- определяет (догадайтесь какой курс?)
\item[Доля пропущенных лекций (6), вес $=-0.67$] опустилась вниз
\end{description}
\center{\textbf{\Large Но и это фигня!}}
\end{frame}

\section{Оценка качества обучения}
\begin{frame}[t]\frametitle{Хорош ли полученный результат}
\begin{itemize}
  \item Отличается ли от КО
  \item Сколько можно выжать из данных
  \item Можно ли верить его компонентам
  \item Воспроизводится ли результат
  \item Насколько можно доверять предсказанию
\end{itemize}
\end{frame}

\subsection{Качество решения}
\begin{frame}[t]\frametitle{По невязке}
В нашем случае невязка -- целевая функция
$$
T(w_0,b_0) = \sum_{(x_i,y_i) \in L} \left\| h_{w_0,b_0}(x_i) - y_i \right\|
$$
\begin{itemize}
  \item Просто посчитать
  \item Рассказывает о работе на тренировочном множестве
  \item {\bf Ничего не говорит о качестве предсказания}
\end{itemize}
\end{frame}

\subsection{Качество предсказания}

\begin{frame}[t]\frametitle{По невязке на другом множестве I}
Можно поделить множество на 2 части, настроить на одной, проверить на другой
$$
  \begin{array}{ll}
    DS = L \cup T, L \cap T = \o \\
    \\
    (w_0,b_0) = \arg\min_(w,b) \sum_{(x_i,y_i) \in L} \left\| x_i^Tw + b - y_i \right\| \\
    T_T = \sum_{(x_i,y_i) \in T} \left\| h_{w_0}(x_i) - y_i \right\|
  \end{array}
$$
\end{frame}

\begin{frame}[t]\frametitle{По невязке на другом множестве II}
\begin{itemize}
  \item Расскажет о качестве предсказания с точностью до деления на $L$ и $T$
  \item Использует меньше данных в обучении
  \item Если исходное множество не показательно, то деление нас не спасет
  \item Можно посмотреть на $T_L$ и $T_T$
\end{itemize}
\end{frame}

\subsection{Стабильность решения}

\begin{frame}[t]\frametitle{По стабильности решения}
Поделим несколько раз и посмотрим на то как меняется $(w_0,b_0)$.
\begin{itemize}
  \item Стабильные компоненты заслуживают веры
  \item Если все нестабильно --- беда-беда
\end{itemize}
\end{frame}

\begin{frame}[t]\frametitle{Немного результатов I}
\scriptsize
$$\begin{array}{ll}
T_L= 176.31 & T_T= 6414.28 \\ 
\sqrt{\frac{T_l(w_0, b_0)}{n}} = 3.55 & \sqrt{\frac{T_t(w_0, b_0)}{n}} = 21.40  \\ 
\\
\multicolumn{2}{l}{
w_0 = -1.80, -0.72, -0.27, 11.70, 5.53, 0.59, -0.13, 0.83, -8.24, 10.15, -1.77, 0.27 %\\
~~ b_0 = 8.71 %\\ 
}\\ 
\\
\\
T_L= 389.83& T_T= 3709.73 \\ 
\sqrt{\frac{T_l(w_0, b_0)}{n}} = 4.79 & \sqrt{\frac{T_t(w_0, b_0)}{n}} = 18.36 \\ 
\\
\multicolumn{2}{l}{
w_0 = -1.96, 1.45, -3.72, -2.57, 1.63, -9.45, 0.76, -1.26, -1.60, 1.83, 2.75, -3.36
~~ b_0 = 10.32 
}\\
\\ 
\\
T_l = 35.94 & T_t = 14181.93 \\ 
\sqrt{\frac{T_l(w_0, b_0)}{n}} = 1.45 & \sqrt{\frac{T_t(w_0, b_0)}{n}} = 35.91 \\ 
\\
\multicolumn{2}{l}{
w_0 = -1.68, 0.62, -22.71, 2.78, 0.97, -0.53, 1.88, 0.70, -0.39, 2.80, -0.46, 1.24
~~b_0 = 10.32
} \\
\\ 
\end{array}$$
\end{frame}

\begin{frame}{Немного результатов II}
А если так делать много раз то можно получить оценки:\\
\vspace{0.5em}
\footnotesize
$$
\begin{array}{rl} 
{w_0}_{1} & 3.09 \pm 50.13 \\
{w_0}_{2} & -8.37 \pm 232.00 \\
{w_0}_{3} & -16.38 \pm 230.82 \\
{w_0}_{4} & -1.11 \pm 94.59 \\
{w_0}_{5} & -0.39 \pm 55.62 \\
{w_0}_{6} & -3.59 \pm 90.58 \\
{w_0}_{7} & -4.86 \pm 124.70 \\
{w_0}_{8} & -10.16 \pm 88.74 \\
{w_0}_{9} & 0.33 \pm 141.19 \\
{w_0}_{10} & 0.54 \pm 45.49 \\
{w_0}_{11} & -2.20 \pm 89.73 \\
{w_0}_{12} & 1.75 \pm 53.67 \\
\\
b_0 & 8.43 \pm 1.57
\end{array}
$$
\end{frame}

\section{Анализ полученных результатов}
\begin{frame}{Можно ли сделать лучше?}
\begin{itemize}
  \item Мало данных или много факторов?
  \begin{itemize}
    \item Все ли факторы одинаково хороши?
    \item Может их можно скомбинировать?
    \item Стоит ли одинаково верить всем факторам?
  \end{itemize}
  \item Может быть в данных что-то нечисто?
  \begin{itemize}
    \item Все ли мы можем объяснить?
    \item А набирали данные правильно?
    \item Не подсматриваем ли мы в ответ?
    % строчки данных, от которых некоторые фичи стреляют
    % данные в лерн содержат результаты про тест, например средние значения по хосту, при делении по запросам
    \item Все ли важные примеры представлены в данных и репрезентативно ли это представление?
  \end{itemize}
\end{itemize}
\end{frame}

\subsection{Не все факторы одинаково полезны}
\begin{frame}{Не все факторы одинаково полезны}
\begin{itemize}
  \item Можем ли мы обойтись без какого-нибудь фактора?
  \item А если фактор преобразовать, может его станет проще использовать?
  \item Если есть похожие факторы, наверное это можно учесть.
  \item Стот ли рассмотреть комбинации нескольких факторов?
  \item Что мы делаем, если фактор посчитать нельзя?
\end{itemize}
\end{frame}

\subsubsection{Информативность фактора}
\begin{frame}[t]\frametitle{Полезна ли информация, которую несет фактор?}

\begin{itemize}
  \item Что мы делаем, если фактор посчитать нельзя?
\end{itemize}
\end{frame}

\subsubsection{Интерпретируемость в рамках модели}
\begin{frame}[t]\frametitle{Как сделать фактор ``съедобным'' для модели?}
Мы использовали линейную модель, в которой, например, категориальные факторы не могут быть эффективно задействованы. Что же делать?
\begin{itemize}
  \item Что мы делаем, если фактор посчитать нельзя?
\end{itemize}
\end{frame}
\subsubsection{Шумность/смещенность}
\subsection{Взаимодействие факторов}
\subsubsection{Корреляция}
\subsubsection{Interaction}
\subsection{Выбросы}
\subsection{Приспособляемость}
\section{Подбор целевой функции}
\subsection{А ту ли задачу мы решаем?}
\subsection{Логистическая регрессия}
\section{Заключение}
% \begin{frame}{Информативные факторы}
% Интерпретируемые
% \begin{itemize}
% % инф, инт
%   \item Баллы ЕГЭ по математике при поступлении
%   \item Литры пива в неделю
%   \item Процент пропусков лекций
%   \item Количество друзей = $\|$ ВК $\cup$ О $\cup$ FB $\|$
% \end{itemize}
% % инф, инт, шум  
% Шумный
% \begin{itemize}
%   \item Текущий средний балл по мнению родителей
% \end{itemize}
% % инф, не инт
% Неинтерпретируемые
% \begin{itemize}
%   \item Пол
%   \item Тип школы
% \end{itemize}
% \end{frame}

% \begin{frame}[t]{Неинформативные}
% Интерпретируемые
% \begin{itemize}
%   \item Расстояние от дома до универа
%   \item Диаметр головы
% \end{itemize}
% Неинтерпретируемые
% \begin{itemize}
%   \item Предпочитаемый ряд в аудитории
%   \item Наличие планшета
% \end{itemize}    
% \end{frame}

\end{document} 
