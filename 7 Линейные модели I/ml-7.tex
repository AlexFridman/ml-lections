\documentclass[14pt, fleqn, xcolor={dvipsnames, table}]{beamer}
\usepackage[T2A]{fontenc}
\usepackage[utf8]{inputenc}
\usepackage[english,russian]{babel}
\usepackage{amssymb,amsfonts,amsmath,mathtext}
\usepackage{cite,enumerate,float,indentfirst}
\usepackage{cancel}

\usepackage{tikz}                   
\usetikzlibrary{shadows}

% \usepackage{enumitem}
% \setitemize{label=\usebeamerfont*{itemize item}%
%   \usebeamercolor[fg]{itemize item}
%   \usebeamertemplate{itemize item}}

\graphicspath{{images/}}

\usetheme{Madrid}
\usecolortheme{seahorse}
\renewcommand{\CancelColor}{\color{red}}

\setbeamercolor{footline}{fg=Blue!50}
\setbeamertemplate{footline}{
  \leavevmode%
  \hbox{%
  \begin{beamercolorbox}[wd=.333333\paperwidth,ht=2.25ex,dp=1ex,center]{}%
    И. Кураленок, Н. Поваров, Яндекс
  \end{beamercolorbox}%
  \begin{beamercolorbox}[wd=.333333\paperwidth,ht=2.25ex,dp=1ex,center]{}%
    Санкт-Петербург, 2013
  \end{beamercolorbox}%
  \begin{beamercolorbox}[wd=.333333\paperwidth,ht=2.25ex,dp=1ex,right]{}%
  Стр. \insertframenumber{} из \inserttotalframenumber \hspace*{2ex}
  \end{beamercolorbox}}%
  \vskip0pt%
}
\newcommand\indentdisplays[1]{%
     \everydisplay{\addtolength\displayindent{#1}%
     \addtolength\displaywidth{-#1}}}
\newcommand{\itemi}{\item[\checkmark]}

\title{Генетические алгоритмы\\\small{}}
\author[]{\small{%
И.~Куралёнок,
Н.~Поваров}}
\date{}

\begin{document}

\begin{frame}
\maketitle
\small
\begin{center}
\vspace{-60pt}
\normalsize {\color{red}Я}ндекс \\
\vspace{80pt}
\footnotesize СПб, 2013
\end{center}
\end{frame}

\section{Содержание}
\begin{frame}{Содержание}
\begin{enumerate}
  \item Генетические алгоритмы
  \item Свойства генетических алгоритмов
  \item Differential evolution
  \item No free lunch theorem
\end{enumerate}
\end{frame}

\begin{frame}{Идея}
Кажется природа решает ту же задачу. \\
$\Rightarrow$ Можно подсмотреть механизмы \\
$\Rightarrow$ Популяция позволяет говорить о "производных" \\
$\Rightarrow$ Внутри популяции можно исследовать зависимости и использовать их \\
$\Rightarrow$ Просто объяснять, но нужен специальный "масонский" язык \\
Отсебятена: генетические алгоритмы скорее язык описания перебора.
\end{frame}

\begin{frame}{Этапы генетического алгоритма}
\begin{enumerate}
  \item Выбрать начальную популяцию
  \item Измерить "приспособленность" особей
  \item Повторить до сходимости
  \begin{enumerate}
    \item Выбрать представителей популяции для размножения
    \item Породить новых особей с помощью скрещивания и мутаций
    \item Измерить приспособленность "детей"
    \item Заменить наименее приспособленных "детьми"
  \end{enumerate}
\end{enumerate}
\end{frame}

\begin{frame}{Выбор начальной позиции}
Как любые переборные методы генетика существенно зависима от начальных позиций. \\
Можно так:
\begin{itemize}
  \item Сводят пространство решений в $[0,1]^n$ и берут равномерно рандомные точки
  \item Сводят к ${0,1}^k$ с помощью бинаризации и берут орты
  \item Делают предварительное сэмплирование
\end{itemize}
\end{frame}

\begin{frame}{Выбор особей для размножения}
Выбор особей - ключевой момент! \\
Можно так:
\begin{itemize}
  \item Выбирать с вероятностью, пропорциональной "приспособленности"
  \item Предварительно делать shuffle "приспособленности"
  \item От "приспособленности" зависит очередность
  \item Можно выбирать равномерно из популяции (по классике нельзя)
  \item Любые комбинации :)
\end{itemize}
\end{frame}

\begin{frame}{Скрещивание}
Хочется сохранить лучшие черты, однако какие лучшие не ясно. Значит устроим случайный процесс!
\begin{itemize}
  \item Потомок имеет связанные "куски" ДНК родителей:
  \begin{itemize}
    \item n-point crossover;
    \item cut'n'splice;
    \item равномерное скрещивание.
  \end{itemize}
  \item Родители влияют на направление развития
  \item Много родителей
\end{itemize}
\end{frame}

\begin{frame}{Мутация}
Использование только скрещивания может привести к вырождению популяции (genetic drift) $Rigtharrow$ мутация.
Мутация очень похожа на фазу выбора новой точки в MCMC/случайном блуждании.
\begin{itemize}
  \item Элементарные строковые преобразования
  \item Инвертирование бита
  \item Случайная мутация компонента
\end{itemize}
\end{frame}

\begin{frame}{Мутация vs Скрещивание}
Мутация - способ "вылезти" из локальных экстремумов, скрещивание - глубину "влезания" в экстремумы. Есть несколько лагерей:
\begin{itemize}
  \item Скрещивание круче, так как быстро бежит. Мутация нужна только, чтобы убедиться, что других хороших направлений нет.
  \item Мутация круче, так как обходит всё. Скрещивание только немного ускоряет процесс.
\end{itemize}
\end{frame}

\begin{frame}{Замещение элементов популяции потомками}
\begin{itemize}
  \item Меняем \% "худших" стариков 
  \item Вводим "penalty" за старость и меняем только если потомок лучше
  \item Старикам здесь не место
\end{itemize}
\end{frame}

\begin{frame}{Основные pros и cons}
\footnotesize
Генетику легко сделать, но сложно понять как она работает. \\
Хорошо:
\begin{itemize}
  \item очень много вариантов отобразить наше понимание область (любой успех легко объяснить :));
  \item легко понять непрофессионалу;
  \item легко программируется;
  \item в отличии от сэмплирования можно применять методы первого порядка, применяя в качестве градиента разности элементов популяции.
\end{itemize}
Плохо:
\begin{itemize}
  \item Когда не работает непонятно за что хвататься;
  \item Результаты зачастую не повторяются на другом множестве(так как незя понять сложность модели);
  \item большое поле для псевдонауки.
\end{itemize}
\end{frame}


\begin{frame}{Алгоритм Differential Evolution}
$$
\arg\max_{\lambda \in \mathbb{R}^n} F(\lambda)
$$


\begin{enumerate}
  \item Выбрать начальную популяцию рандомом
  \item До сходимости, для каждого элемента популяции $x \in P$:
  \begin{enumerate}
    \item выбрать $a, b, c \in P$ чтобы все отличались;
    \item $k \sim U(1..n)$;
    \item $y = (y_i)$ для каждого $i$
    \begin{enumerate}
      \item $r \sim U((0,1))$
      \item $y_i = \left\{ \begin{array}{ll} a_i + F(b_i - c_i), i =k | r < C \\ y_i = x_i \\ \end{array} \right. $
    \end{enumerate}
    \item Заменить старый элемент новым, если новые лучше.
  \end{enumerate}
  \item Выбрать лучшего представителя получившейся популяции.
\end{enumerate}
\end{frame}

\section{No free lunch theorem}
\begin{frame}{Постановка вопроса}
\textit{Целых 2 занятия мы обходим пространство решений. Может быть есть лучший из всех методов обхода?}
В качестве аналогии:
\begin{description}
  \item[Ресторан:] способ обхода пространства решений
  \item[Блюдо:] конкретная задача, требующая оптимизации
  \item[Цена:] значение целевой функции на обойденной части пространства
\end{description}
\begin{theorem}[No free lunch theorem]
Все рестораны в среднем одинаково дороги
\end{theorem}
\end{frame}

\begin{frame}{NFL: формальная формулировка}
$$\begin{array}{rl}
d_m =& \{(d^x_m(1), d^y_m(1)), \ldots, (d^x_m(m), d^y_m(m))\} \\
f: & \mathcal{X} \to \mathcal{Y} \\
\mathcal{F}= & \mathcal{Y}^\mathcal{X} \\
p(f) = & \frac{1}{\mathcal{F}} ёё
\end{array}$$

\begin{theorem}[David Wolpert and William G. Macready (1997)]
Для любых двух способов обхода $a_1$ и $a_2$:
$$\begin{array}{rl}
\sum_f p(d^y_m|f,m,a_1) = \sum_f p(d^y_m|f,m,a_2) \\
\sum_f p(d^y_m|f_0,M,m,a_1) = \sum_f p(d^y_m|f_0,M,m,a_2)
\end{array}$$
\end{theorem}
\end{frame}

\begin{frame}{NFL: следствия}
\begin{itemize}
  \item Нам всем хватит работы :)
  \item Полоса белая, полоса черная
  \item Нужно искать близкие задачи
\end{itemize}
\end{frame}


\begin{frame}{Домашнее задание}
Кто согласен с основным утверждением лекции, что генетика это только альтернативный способ описания перебора:
\begin{itemize}
  \item Фитим ln(x) 
  \item Датасет старый
  \item Полином ищем генетикой
\end{itemize}
Кто не согласен:
\begin{itemize}
  \item Доводы
  \item Примеры
\end{itemize}
\end{frame}

\end{document}
